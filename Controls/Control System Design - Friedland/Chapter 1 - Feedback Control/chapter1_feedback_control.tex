\documentclass[10pt]{article}

\usepackage[utf8]{inputenc}
\usepackage{amsmath,amsthm,amssymb}
\usepackage{mathrsfs}
\usepackage{amsmath}
\usepackage{epsfig}
\usepackage{color}
\usepackage{indentfirst}
\usepackage{enumitem}
\usepackage[marginparwidth=1in]{geometry}
\usepackage{marginnote}
\usepackage{listings}
\usepackage{xcolor}
\usepackage{graphicx}
\usepackage{latexsym,amsfonts,amssymb,amsthm,amsmath}
\usepackage{matlab-prettifier}
\usepackage[
    backend=biber,
    style=alphabetic,
    sorting=ynt
]{biblatex}
\addbibresource{Reference.bib}

\newcommand{\N}{\mathbb{N}}
\newcommand{\Z}{\mathbb{Z}}

\newenvironment{theorem}[2][Theorem]{
    \begin{trivlist}
\item[\hskip \labelsep {\bfseries #1}\hskip \labelsep {\bfseries #2.}]}{
\end{trivlist}}
\newenvironment{lemma}[2][Lemma]{
    \begin{trivlist}
\item[\hskip \labelsep {\bfseries #1}\hskip \labelsep {\bfseries #2.}]}{
\end{trivlist}}
\newenvironment{exercise}[2][Exercise]{
    \begin{trivlist}
\item[\hskip \labelsep {\bfseries #1}\hskip \labelsep {\bfseries #2.}]}{
\end{trivlist}}
\newenvironment{problem}[2][Problem]{
    \begin{trivlist}
\item[\hskip \labelsep {\bfseries #1}\hskip \labelsep {\bfseries #2.}]}{
\end{trivlist}}
\newenvironment{question}[2][Question]{
    \begin{trivlist}
\item[\hskip \labelsep {\bfseries #1}\hskip \labelsep {\bfseries #2.}]}{
\end{trivlist}}
\newenvironment{corollary}[2][Corollary]{
    \begin{trivlist}
\item[\hskip \labelsep {\bfseries #1}\hskip \labelsep {\bfseries #2.}]}{
\end{trivlist}}

\newlist{indentlist}{enumerate}{2}
\setlist[indentlist, 1]{label=(\alph*), listparindent=\parindent}
\setlist[indentlist, 2]{label=\roman*., listparindent=\parindent}

\newlist{noindentlist}{enumerate}{2}
\setlist[noindentlist, 1]{label=(\alph*), listparindent=\parindent, leftmargin=6mm}
\setlist[noindentlist, 2]{label=\roman*., listparindent=\parindent}

\newenvironment{solution}{
\begin{proof}[Solution]}{
\end{proof}}

\newcommand{\rank}{{\rm rank\;}}
\newcommand{\image}{{\rm Im\;}}
\newcommand{\stack}{{\rm stack\;}}
\newcommand{\Span}{{\rm span\;}}
\newcommand{\diag}{{\rm diag\;}}
\newcommand{\col}{{\rm col\;}}
\newcommand{\cp}{{\rm cp\;}}
\newcommand{\R}{\mathbb{R}}
\newtheorem{remark}{Remark}
\newtheorem{proposition}{Proposition}
\newtheorem{property}{Property}
\newtheorem{conjecture}{Conjecture}
\newtheorem{assumption}{Assumption}
\newtheorem{example}{Example}
\def\n{{\bf n}}
\def\m{{\bf m}}
\def\r{{\bf r}}
\def\c{{\bf c}}
\def\scr#1{{\mathcal #1}}
\def\rep#1{(\ref{#1})}
\def\eq#1{
    \begin{equation}#1
\end{equation}}
\def\qed{ \rule{.1in}{.1in}}

\def\matt#1{
    \begin{bmatrix}#1
\end{bmatrix}}
\newcommand{\vecb}[1]{\boldsymbol{#1}}
\newcommand{\dvecb}[1]{\dot{\boldsymbol{#1}}}
\newcommand{\ddvecb}[1]{\ddot{\boldsymbol{#1}}}
\newcommand{\unitv}[1]{\hat{\boldsymbol{#1}}}
\newcommand{\vecbt}[1]{\Tilde{\boldsymbol{#1}}}
\newcommand{\vecbb}[1]{\Bar{\boldsymbol{#1}}}
\newcommand{\norm}[2]{\left\Vert {#1} \right\Vert_{#2}}
\newenvironment{aequation}
{
    \begin{equation}
    \begin{aligned}}
        {
        \end{aligned}
\end{equation}}

\definecolor{codegreen}{rgb}{0,0.6,0}
\definecolor{codegray}{rgb}{0.5,0.5,0.5}
\definecolor{codepurple}{rgb}{0.58,0,0.82}
\definecolor{backcolour}{rgb}{0.95,0.95,0.92}
\lstdefinestyle{mystyle}{
    backgroundcolor=\color{backcolour},
    commentstyle=\color{codegreen},
    keywordstyle=\color{magenta},
    numberstyle=\tiny\color{codegray},
    stringstyle=\color{codepurple},
    basicstyle=\ttfamily\footnotesize,
    breakatwhitespace=false,
    breaklines=true,
    captionpos=b,
    keepspaces=true,
    numbers=left,
    numbersep=5pt,
    showspaces=false,
    showstringspaces=false,
    showtabs=false,
    tabsize=2,
}
\lstset{style=mystyle}

\begin{document}

% --------------------------------------------------------------
%                         Start here
% --------------------------------------------------------------

\title{Chapter 1 - Feedback Control}
\author{Joseph Le}

\maketitle

\section{The Mechanism of Feedback}

$H$ is the process we wish to control, $u$ is the control input to $H$, and $y$ is the output of the  process. Open-loop control is controlling the process to produce a desired and known output, $\bar u$ produces $\bar y$.

Feedback often uses the error from a desired output

\begin{equation}
    \label{outputerror}
    e = \bar{y} - y
\end{equation}

In order to compute the desired control signal $u$. This is amplified by a value $K$. The output is represented as

\begin{equation}
    \label{output}
    y = H u
\end{equation}

and the control signal is represented as

\begin{equation}
    u = K e
\end{equation}

Then combining the previous three equations results in the following relation

$$ y = HKe = HK(\bar y - y) $$

\noindent and solving for $y$

\begin{equation}
    y = \frac{HK}{1 + HK}\bar y
\end{equation}

Here, $y \neq \bar y$, but a large enough $K$ results in $HK \ll 1$ so

\begin{equation}
    y \simeq \bar y
\end{equation}

This holds irregardless of process, time variance, or $\bar y$. Though, the relationship in eq.\ref{output}, is not valid for all systems (i.e. nonlinear systems). Suppose the output $y$ of $H$ is a delayed replica of the input signal $u$ delayed by the amount $\tau$ resulting in

\begin{equation}
    \label{outputdelay}
    y(t) = u(t-\tau)
\end{equation}

Assuming equations \ref{outputerror} and \ref{output} $\forall t$, then

\begin{equation}
    \label{inputequation}
    u(t-\tau) = Ke(t-\tau) = K\left[ \bar y(t-\tau) - y(t-\tau) \right]
\end{equation}

Substituting equation \ref{inputequation} into \ref{outputdelay} results in

\begin{equation}
    y(t) = K \left[ \bar y(t-\tau) - y(t-\tau) \right]
    \label{output2}
\end{equation}

\noindent this is a difference equation that describes how $y(t)$ changes over time and is often used for discrete-time systems. To solve this, suppose the desired output is a unit step:

\begin{equation}
    \bar y(t) =
    \begin{cases}
        0 & \text{for } t < 0 \\
        1 & \text{for } t > 0
    \end{cases}
\end{equation}

\noindent and $y(t) = 0$ for $t < 0$. Then, by equation \ref{output2}, it can be seen that there is zero output for the first $\tau$ units of time. The input is

$$
u(t) = K(1-0) = K \quad \text{for } 0<t<\tau
$$

After $\tau$ time units, the input is

$$
u(t) = K(1-K) = K - K^2 \quad \text{for } \tau < t < 2\tau
$$

\noindent For the next $\tau$ time units, $y(t)$ is equal to this (i.e. for $2\tau < t < 3\tau$). Continuing this results in

\begin{equation}
    y(t) = K - K^2 + K^3 + \cdots + (-1)^{n-1}K^{n-1} \quad \text{for } n\tau < t < (n+1)\tau
    \label{outputn}
\end{equation}

If $K < 1$ then equation \ref{outputn} implies that $y(t)$ converges to a limit

\begin{equation}
    \lim_{t \rightarrow \infty} y(t) = K - K^2 + K^3 - K^4 + \cdots = \frac{K}{1 + K}
\end{equation}

\noindent If $K = 1$ then it bounces between $0$ and $1$, if $K>1$, then it diverges in amplitude. As $K$ approaches $1$, the output is half the desired, but we can multiply the desired output by $1+K$ before the comparison.

This result shows that the gain cannot but arbitrarily large for all systems (especially those with delays, even arbitrarily small delays). Most real systems include delays but control inputs are not going to be copies of the output. The amplifier, $K$, changes the shape of the signal and is called a "compensator".

\section{Feedback Control Engineering}





\section{Control Theory Background}

\section{Scope and Organization of This Book}

% \printbibliography
\end{document}