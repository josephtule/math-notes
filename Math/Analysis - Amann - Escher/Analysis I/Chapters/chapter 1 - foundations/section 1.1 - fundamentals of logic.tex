\section{Fundamentals of Logic}

Symbolic logic\index{logic} is about \textbf{statements}\index{logic!statements} which can be either true (T) or false (F), not both or in between.
A statement $A$ can have a \textbf{negation}\index{negation} $\neg A$ (`not $A$') which is defined as: $\neg A$ true if $A$ false.
A truth table can be used to show this:

$$
\begin{array}{|c||c|c|}
    \hline
    A & T & F \\
    \hline{}
    \neg A & F & T \\
    \hline
\end{array}
$$

Two statements $A$ and $B$ can be combined using \textbf{conjunction}\index{conjunction} or \textbf{disjunction}\index{disjunction} to make new ones.
The statement $A \land B$ (`$A$ and $B$') is true when both $A$ and $B$ are true.
The statement $A \lor B$ (`$A$ or $B$') is true when either $A$ or $B$ or both are true (inclusive or) and is false only when both are false.
Refer to the following truth table:

$$
\begin{array}{|c|c||c|c|}
    \hline
    A & B & A \land B & A \lor B \\
    \hline
    T & T & T & T \\
    T & F & F & T \\
    F & T & F & T \\
    F & F & F & F \\
    \hline
\end{array}
$$

If $E(x)$ is a statement where $x$ is replaced by an object (member, thing) of a specified class (collection, universe), then $E$ is a \textbf{property}\index{property}.
`$x$ has property $E$' is equivalent to `$E(x)$ is true'.
If $x$ is a member (an \textbf{element}\index{element}) of a class $X$ we write $x \in X$ otherwise $x \not\in X$.
Then,

$$
\{x \in X : E(x)\}
$$

\noindent is the class of all elements $x$ of the collection $X$ that have the property $E$.

The \textbf{quantifier}\index{quantifier} $\exists$ denotes existence and is read `there exists'.
The expression

$$
\exists x \in X : E(x)
$$

\noindent says `There is (at least) one object $x$ in (the class) $X$ which has the property $E$'.
The unique quantifier $\exists!$ says that there is only one (a unique) object.

The quantifier $\forall$ denotes `for all' or `for each' object in a collection.
The expression

\begin{equation}
    \forall x \in X : E(x)
\end{equation}

\noindent says `For each $x \in X$ the statement $E(x)$ is true'.
This can also be written as

\begin{equation}
    E(x), \quad \forall x \in X
\end{equation}

\noindent which says `The property $E(x)$ is true for all $x$ in $X$'.
Sometimes the quantifier is left out

\begin{equation}
    E(x), \quad x \in X
\end{equation}

The symbol $:=$ means `is defined by'.
Thus,

$$
a := b
$$

\noindent says `the object (or symbol) $a$ is defined by the object (or expression) $b$'.
Of course $a = b$ means $a$ and $b$ are equal.

\begin{example}\label{ex:1.1} Let $A$ and $B$ be statements, $X$ and $Y$ be classes of objects, and $E$ is a property. Truth tables can verify the following statements:

    \begin{enumerate}[label=(\alph*)]
        \item $\neg\neg A := \neg(\neg A) = A$
            $$
            \begin{array}{|c||c|c|}
                \hline
                A & \neg A & \neg\neg(A) \\
                \hline
                T & F & T \\
                F & T & F \\
                \hline
            \end{array}
            $$
        \item $\neg(A \land B) = (\neg A) \lor (\neg B)$ (de Morgan's Law 1)
            $$
            \begin{array}{|c|c||c|c|c|}
                \hline
                A & B & A \land B & \neg (A \land B) & (\neg A) \lor (\neg B) \\
                \hline
                T & T & T         & F                & F \\
                T & F & F         & T                & T \\
                F & T & F         & T                & T \\
                F & F & F         & T                & T \\
                \hline
            \end{array}
            $$
        \item \label{ex:1.1(c)} $\neg(A \lor B) = (\neg A) \land (\neg B)$ (De Morgan's Law 2)
            $$
            \begin{array}{|c|c||c|c|c|}
                \hline
                A & B & A \lor B & \neg (A \lor B) & (\neg A) \land(\neg B) \\
                \hline
                T & T & T         & F                & F \\
                T & F & T         & F                & F \\
                F & T & T         & F                & F \\
                F & F & F         & T                & T \\
                \hline
            \end{array}
            $$
        \item $\neg (\forall x \in X : E(x)) = (\exists x \in X : \neg E(x))$ (for example: the negation of `everyone wears glasses' is `at least one person does not wear glasses')
        \item $\neg (\exists x \in X : E(x)) = (\forall x \in X : \neg E(x))$ (for example: the negation of `there is at least one car on the road that is red' is `there are no cars on the road that is read')
        \item $\neg (\forall x \in X : (\exists y \in Y : E(x,y))) = (\exists x \in X : (\forall y \in Y : \neg E(x,y)))$ (for example: the negation of `every person has something in at least one pocket' is `there is at least one person that has whose pockets are all empty'. Here X is the collection of people, Y is the collection of pockets on a person, and E is the property that the pocket is occupied and depends on the person and the pocket.)
        \item $\neg (\exists x \in X : (\forall y \in Y : E(x,y))) = (\forall x \in X : (\exists y \in Y : \neg E(x,y)))$ (for example: the negation of `there is at least one car with every window open' is `in the collection of all cars there is at least one window that is closed')
    \end{enumerate}
\end{example}

\begin{remark}
    \begin{enumerate}[label=(\alph*)]
        \item Parenthesis keep the statements exact but aren't always used; similarly, the membership symbol is not always used. For example: $\forall x \exists y : E(x,y)$ is still valid and says `For all $x$ there is at least one $y$ such that $E(x,y)$ is true', thus $y$ depends on $x$. Another example: $\exists x \forall y : E(x,y)$ which says `there is at least one $x$ with every $y$ such that $E(x,y)$ is true' it is sufficient to find one $y$ which is true for all $x$. For example, if $E(x,y)$ is the statement `reader $x$ of this book find the concept of $y$ to be trivial' then the first statement is: `Each reader of this book finds at least one concept that is trivial' and the second statement is `there is at least one statement that every reader finds trivial.'
        \item The quantifiers $\exists$ and $\forall$ as well as the logical `and' and `or' are mechanically interchanged in negation without changing order while the statements are negated.
    \end{enumerate}
\end{remark}

Let $A$ and $B$ be statements, the \textbf{implication}\index{implication} $A \implies B$, (`$A$ implies $B$') is defined as:

\begin{equation}\label{eqn:implies definition}
    (A \implies B) := (\neg A) \lor B
\end{equation}

Thus $A \implies B$ is false if $A$ is true and $B$ is false, and true in all other cases. In other words, it is true when $A$ and $B$ are both true, or when $A$ is false (independent of whether $B$ is true or false) meaning a true statement cannot imply a false statement, also a false statement implies any statement. Common to say `To prove $B$ it \textbf{suffices} to prove $A$' or `$B$ is \textbf{necessary} for $A$ to be true', in other words, $A$ is a \textbf{sufficient condition}\index{sufficient condition} for $B$ and $B$ is a \textbf{necessary condition}\index{necessary condition} for $A$.

The \textbf{equivalence}\index{equivalence} $A \iff B$ (`$A$ and $B$ are equivalent') of the statements is defined by:

$$
(A \iff B) := (A \implies B) \land (B \implies A)
$$

$$
\begin{array}{|c|c||c|c|c|}
    \hline
    A & B & A \implies B & B \implies A & A \iff B \\
    \hline
    T & T & T & T & T \\
    T & F & F & T & F \\
    F & T & T & F & F \\
    F & F & T & T & T \\
    \hline
\end{array}
$$

\noindent $B \implies A$ is the \textbf{converse}\index{converse} of $A \implies B$, and $A$ is a \textbf{necessary and sufficient}\index{necesssary and sufficient} condition for B (or vice versa), another way to express this is to say `$A$ is true \textbf{if and only if}\index{if and only if (iff)} (iff) $B$ is true'.

A fundamental observation:

\begin{equation}
    (A \implies B) \iff (\neg B \implies \neg A)
\end{equation}

$$
\begin{array}{|c|c||c|c|c|c|}
    \hline
    A & B & \neg A & \neg B & A \implies B & \neg B \implies \neg A \\
    \hline
    T & T & F & F & T & T \\
    T & F & F & T & F & F \\
    F & T & T & F & T & T \\
    F & F & T & T & T & T \\
    \hline
\end{array}
$$

\noindent This follows directly from Equation \ref{eqn:implies definition} and is called the \textbf{contrapositive}\index{contrapositive} of the statement $A \implies B$.

For example, if $A$ is `There are clouds in the sky' and $B$ is `It is raining', then $B \implies A$ is the statement `If it is raining, then there are clouds in the sky'. Its contrapositive is `If there are no clouds in the sky, then it is not raining'.

If $B \implies A$ is true it does not, in general, follow that $\neg B \implies \neg A$ is true, for example, even if `It is not raining' it is possible that `There are clouds in the sky'.

The following defines $A$ is true whenever $B$ is true and is read as `$A$ is true, by definition, if $B$ is true':

$$
A :\iff B
$$

\medskip

In math, a true statement is often called a \textbf{proposition}\index{proposition}, \textbf{theorem}\index{theorem}, \textbf{lemma}\index{lemma}, or \textbf{corollary}\index{corollary}. Typically, propositions are of the form $A \implies B$ and since the statement is automatically true if $A$ is false (see last two lines of the truth table), to prove the statement true, suppose $A$ is true then show that $B$ is also true.

Proofs can be done directly or `by contradiction'. Directly, one can use the fact that 

\begin{equation} \label{eqn:a implies c implies b}
    (A \implies C) \land (C \implies B) \implies (A \implies B)
\end{equation}

If $A \implies C$ and $C \implies B$ are each separately already known to be true, then by Equation \ref{eqn:a implies c implies b} $A \implies B$ is also true. 

A proof by \textbf{contradiction}\index{contradiction} supposes that $B$ is false. The one proves, using the assumption that $A$ is true, a statement $C$ which is already known to be false. It follows from this `contradiction' that $\neg B$ cannot be true, and hence $B$ is true.

Sometimes easier to prove contrapositive.

\subsection*{Exercises}

\begin{enumerate}
    \item TODO: do this
    \item TODO: do this
\end{enumerate}