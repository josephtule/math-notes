% !TEX root = ./main.tex
% --------------------------------------------------------------
%                  ENCODING AND GEOMETRY
% --------------------------------------------------------------
\usepackage[utf8]{inputenc}
\usepackage[T1]{fontenc}
% \usepackage{indentfirst} % TODO: keep or remove?
\usepackage{geometry}
\geometry{margin=1in}
% \geometry{a4paper, total={6in, 8in}}
\setlength{\parskip}{0.25em} % skips after each paragraph

% --------------------------------------------------------------
%                       MATH PACKAGES
% --------------------------------------------------------------
\usepackage{amsmath, amssymb, amsfonts, amsthm, mathtools}
\usepackage{mathrsfs} % cannot be used with the following line (standard math font)
% \usepackage{newtxtext,newtxmath} % cannot be used with the previous line (springer math font)
\usepackage{esint} % extended integrals
\usepackage{epsdice}
\newcommand\dice[1]{\vcenter{\hbox{\epsdice{#1}}}}

% --------------------------------------------------------------
%                       GRAPHICS & COLOR
% --------------------------------------------------------------
\usepackage{graphicx}
\usepackage{xcolor}
\usepackage{tikz}
\usetikzlibrary{positioning,decorations.pathmorphing,arrows,calc,angles,quotes}
\usepackage{pgfplots}
\pgfplotsset{compat=1.18}
\usepackage{diagbox}

% --------------------------------------------------------------
%                      LISTS AND STYLING
% --------------------------------------------------------------
\usepackage{enumitem}
\setlist{leftmargin=2em}
\usepackage[english]{babel}
\usepackage[autostyle, english=american]{csquotes}
\MakeOuterQuote{"}
% NOTE: use ``'' over "", less errors, but "" ensures quotes are closed

% --------------------------------------------------------------
%                   HYPERLINKS AND REFERENCES
% --------------------------------------------------------------
\usepackage{hyperref}
\usepackage{cleveref}
\hypersetup{
    colorlinks=true,
    linkcolor=black,       % Internal links (e.g., TOC, equations)
    citecolor=black,       % Citation links
    urlcolor=blue          % External links
}
\usepackage{makeidx}

% --------------------------------------------------------------
%                    BIBLIOGRAPHY SETUP
% --------------------------------------------------------------
\usepackage[
    backend=biber,
    style=numeric,
    sorting=nyt
]{biblatex}
\addbibresource{Reference.bib}

% --------------------------------------------------------------
%                   THEOREM ENVIRONMENTS
% --------------------------------------------------------------
\theoremstyle{plain}
% NOTE: change depending on needs (chapter or section)
% \newtheorem{theorem}{Theorem}[chapter]
\newtheorem{theorem}{Theorem}[section]
\newtheorem{lemma}[theorem]{Lemma}
\newtheorem{corollary}[theorem]{Corollary}
\newtheorem{proposition}[theorem]{Proposition}

\theoremstyle{definition}
\newtheorem{definition}[theorem]{Definition}
\newtheorem{example}[theorem]{Example}
\newtheorem{fact}[theorem]{Fact}
\newtheorem{exercise}[theorem]{Exercise}
\newtheorem{remark}[theorem]{Remark}
\newtheorem{conjecture}[theorem]{Conjecture}

% NOTE: \newtheorem{type}{Name}[number]

% NOTE: choose remark style
% \theoremstyle{remark}
% \newtheorem*{remark}{Remark}

% --------------------------------------------------------------
%                        CUSTOM COMMANDS
% --------------------------------------------------------------
% Typeface
\newcommand{\tbf}[1]{\textbf{#1}}
\newcommand{\tif}[1]{\textit{#1}}
\newcommand{\tef}[1]{\emph{#1}}

% Sets
\newcommand{\R}{\mathbb{R}}
\newcommand{\N}{\mathbb{N}}
\newcommand{\Z}{\mathbb{Z}}
\newcommand{\Q}{\mathbb{Q}}
\newcommand{\C}{\mathbb{C}}
\newcommand{\setin}{\subseteq}
\newcommand{\supin}{\supseteq}
\newcommand{\vno}{\varnothing}
\newcommand{\converse}{\Longleftarrow}

% Vectors and formatting
\newcommand{\vecb}[1]{\boldsymbol{#1}}
\newcommand{\unitv}[1]{\hat{\boldsymbol{#1}}}
\newcommand{\norm}[2]{\left\Vert {#1} \right\Vert_{#2}}

% Operators
\newcommand{\Span}{\operatorname{span}}
\newcommand{\rank}{\operatorname{rank}}
\newcommand{\diag}{\operatorname{diag}}
\newcommand{\image}{\operatorname{Im}}

% Matrices
\newcommand{\matt}[1]{
    \begin{bmatrix}#1
\end{bmatrix}}

% Align environment for equations
\newenvironment{aequation}
{
    \begin{equation}
        \begin{aligned}
        }
        {
        \end{aligned}
    \end{equation}
}

\newcommand{\Interval}[4]{%
    \tikz{%
        \coordinate [label={center:$#1$},label=below:$\rule{0pt}{.35cm}#2$] (a) at (0,0);
        \coordinate [label={center:$#3$},label=below:$\rule{0pt}{.35cm}#4$] (b) at (1.6,0);
        \draw[-{latex},] decorate[decoration=zigzag] {(a)--(b)} (-.4,0)--(2,0);
    }
}

\newcommand\interval[4]{%
    \begin{tikzpicture}[text height=1ex]%
        \draw[<-] (0,0) -- (1,0);
        \draw[{#3-#4}] (1,0) node[label=below:{#1}] {} -- (2,0) node[label=below:{#2}] {};
        \draw[->] (2,0) -- (3,0);
    \end{tikzpicture}
}

\newcommand\intervalinline[4]{%
    \raisebox{10pt}{% Adjust vertical alignment if needed
        \begin{tikzpicture}[baseline={(0,0.5ex)}, text height=1ex, text depth=0.25ex]
            \draw[<-] (0,0) -- (1,0);
            \draw[{#3}-{#4}] (1,0) node[label=below:{#1}] {} -- (2,0) node[label=below:{#2}] {};
            \draw[->] (2,0) -- (3,0);
        \end{tikzpicture}%
    }%
}

\makeindex
