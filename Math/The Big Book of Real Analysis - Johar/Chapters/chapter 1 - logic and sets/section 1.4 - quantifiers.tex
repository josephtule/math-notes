\section{Quantifiers}

\begin{remark}
    This is left empty.
\end{remark}

\begin{example}
    This is left empty.
\end{example}

\begin{definition}[Universal, Existential Quantifiers]
    This is left empty.
\end{definition}

\begin{remark}
    We make several remarks here:

    \begin{enumerate}
        \item Universal Quantifier:
            \begin{enumerate}[label=(\alph*)]
                \item Similar to the "and" connective, require all statements involved to be true for the compound statement to be true.
                \item $\forall$ read as "for all" or "for every" or "for each" or "for any" or "for arbitrary".
                \item $(\forall x \in X)$, $P(x)$ read as "For all $x \in X$, $P(x)$ is true" or "$P(x)$ is true for all $x \in X$".
            \end{enumerate}
        \item Existential Quantifier:
            \begin{enumerate}[label=(\alph*)]
                \item Similar to the "or" connective, require at least one statement involved to be true for the compound statement to be true.
                \item $\exists$ read as "there exists" or "there are some" or "there is at least one" or "for some" or "for at least one".
                \item $(\exists x \in X): P(x)$ read as "There exists an $x \in X$ such that $P(x)$ is true" or "$P(x)$ is true for some $x \in X$".
            \end{enumerate}
        \item The colon : used in similar manner to set builder notation. Read as "such that". Not necessary in the universal quantifier example, same for the comma in the existential quantifier example.
        \item Most of the time, quantifier parentheses not used.
    \end{enumerate}
\end{remark}

\begin{example}

    \begin{enumerate}
        \item Suppose $X$ and $Y$ are sets.
            \begin{enumerate}[label=(\alph*)]
                \item $X \subseteq Y \iff (\forall x \in X), P(x)$
                \item $X \cap Y \neq \varnothing \iff (\exists x \in X): P(x)$
                \item $X \cap Y = \varnothing  \iff (\forall x \in X), \neg P(x)$
            \end{enumerate}

        \item Let $X$ be the set of months $X = \{\text{Jan, Feb, ..., Dec}\}$ and $P(x)$ is "The month $x$ has 30 days in it"
            \begin{enumerate}[label=(\alph*)]
                \item $(\forall x \in X), P(x)$ reads as "For every month $x$ in $X$, the month $x$ has 30 days". This compound statement is false, because February has 28 or 29 days and some months have 31 days. Logically, and used in many proofs, there can be found at least one $x \in X$ that does not satisfy $P(x)$.
                \item $(\exists x \in X): P(x)$ reads as "There exists a month $x$ in $X$ such that the month $x$ has 30 days in it". This is true because January has 30 days, thus the compound statement is true.
            \end{enumerate}

        \item Let $\Gamma$ be the set of all polygons. For each $\gamma \in \Gamma$, we define:
            $$
            P(\gamma): \gamma \text{ is a square}, \quad \text{ and } \quad Q(\gamma): \gamma \text{ is a rectangle}
            $$
            \begin{enumerate}[label=(\alph*)]
                \item $(\forall \gamma \in \Gamma), P(\gamma)$ is false because it reads as "For any polygon $\gamma \in \Gamma$, it is a square"; but a pentagon is a polygon and it is not a square.
                \item $(\exists \gamma \in \Gamma) : P(\gamma)$ reads as "There exists a polygon $\gamma \in \Gamma$, such that $\gamma$ is square" which is true because squares are polygons.
                \item $(\forall \gamma \in \Gamma), (P(\gamma) \implies Q(\gamma))$ is read as "For all polygons $\gamma \in \Gamma$, if $\gamma$ is a square then it is a rectangle" which is true because all squares are rectangles.
            \end{enumerate}

        \item Consider the set of birds $B$ and family of statements $\{P(b) : b \in B\}$ where $P(b)$ is "The bird $b$ can fly". This statement is false because there are birds that cannot fly. Therefore $(\forall b \in B), P(b)$ which says "For each bird $b$ in $B$ it can fly" is false. The negation $\neg ((\forall b \in B),P(b))$ must be true. The negation says "There is at least one bird $b$ such that the bird $b$ cannot fly" or $(\exists b \in B): \neg P(b)$. So we have the equivalence:
            $$
            \neg((\forall b \in B), P(b)) \equiv (\exists b \in B):\neg P(b)
            $$
            \label{ex:quantifiers4}

        \item Let $X$ and $Y$ be sets. For each $x \in X$, define $P(x)$ to be the statement $x \in Y$ and $Q$ be the statement "$X \cap Y \neq \varnothing$".
            \begin{enumerate}[label=(\alph*)]
                \item $Q$ says that there is at least one element in both $X$ and $Y$ and $Q \equiv (\exists x \in X): P(x)$.
                \item The negation of $Q$ namely $\neg Q$ is $X \cap Y = \varnothing$ and $\neg Q \equiv (\forall x \in X), \neg P(x)$.
            \end{enumerate}
            Thus, we have the equivalence: \label{ex:quantifiers5}
            $$
            (\forall x \in X), \neg P(x) \equiv \neg Q \equiv \neq((\exists x \in X) : P(x))
            $$
    \end{enumerate}
    \label{ex:quantifiers}
\end{example}

In examples \ref{ex:quantifiers}(\ref{ex:quantifiers4}) and (\ref{ex:quantifiers5}) the following rules hold:

\begin{enumerate}
    \item $\neg((\exists x \in X): P(x)) \equiv (\forall x \in X), \neg P(x)$
    \item $\neg((\forall x \in X), P(x)) \equiv (\exists x \in X): \neg P(x)$
\end{enumerate}

\noindent where the negation of a compound quantifier statement results in the flipping of the quantifier and negation of the specified statement. These are called De Morgan's laws in formal logic.

\begin{example}
    Define two sets $\Gamma$ and $\Delta$ where $\Gamma$ is the set of letters in Latin and $\Delta$ is the set of all words in \textit{The Oxford English Dictionary}. 
    Define a mathematical statement that depends on two variables $(\gamma,\delta) \in \Gamma \times \Delta$ which says "The word $\delta$ begins with the letter $\gamma$" as $P(\gamma,\delta)$

    \begin{enumerate}
        \item 
    \end{enumerate}

\end{example}