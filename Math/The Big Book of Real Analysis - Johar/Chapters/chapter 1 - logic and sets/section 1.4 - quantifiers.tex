\section{Quantifiers}

\begin{remark}
    This is left empty.
\end{remark}

\begin{example}
    This is left empty.
\end{example}

\begin{definition}[Universal, Existential Quantifiers]
    This is left empty.
\end{definition}

\begin{remark}
    We make several remarks here:

    \begin{enumerate}
        \item Universal Quantifier:
            \begin{enumerate}[label=(\alph*)]
                \item Similar to the "and" connective, require all statements involved to be true for the compound statement to be true.
                \item $\forall$ read as "for all" or "for every" or "for each" or "for any" or "for arbitrary".
                \item $(\forall x \in X)$, $P(x)$ read as "For all $x \in X$, $P(x)$ is true" or "$P(x)$ is true for all $x \in X$".
            \end{enumerate}
        \item Existential Quantifier:
            \begin{enumerate}[label=(\alph*)]
                \item Similar to the "or" connective, require at least one statement involved to be true for the compound statement to be true.
                \item $\exists$ read as "there exists" or "there are some" or "there is at least one" or "for some" or "for at least one".
                \item $(\exists x \in X): P(x)$ read as "There exists an $x \in X$ such that $P(x)$ is true" or "$P(x)$ is true for some $x \in X$".
            \end{enumerate}
        \item The colon : used in similar manner to set builder notation. Read as "such that". Not necessary in the universal quantifier example, same for the comma in the existential quantifier example.
        \item Most of the time, quantifier parentheses not used.
    \end{enumerate}
\end{remark}

\begin{example}

    \begin{enumerate}
        \item Suppose $X$ and $Y$ are sets.
            \begin{enumerate}[label=(\alph*)]
                \item $X \subseteq Y \iff (\forall x \in X), P(x)$
                \item $X \cap Y \neq \varnothing \iff (\exists x \in X): P(x)$
                \item $X \cap Y = \varnothing  \iff (\forall x \in X), \neg P(x)$
            \end{enumerate}

        \item Let $X$ be the set of months $X = \{\text{Jan, Feb, ..., Dec}\}$ and $P(x)$ is "The month $x$ has 30 days in it"
            \begin{enumerate}[label=(\alph*)]
                \item $(\forall x \in X), P(x)$ reads as "For every month $x$ in $X$, the month $x$ has 30 days". This compound statement is false, because February has 28 or 29 days and some months have 31 days. Logically, and used in many proofs, there can be found at least one $x \in X$ that does not satisfy $P(x)$.
                \item $(\exists x \in X): P(x)$ reads as "There exists a month $x$ in $X$ such that the month $x$ has 30 days in it". This is true because January has 30 days, thus the compound statement is true.
            \end{enumerate}

        \item Let $\Gamma$ be the set of all polygons. For each $\gamma \in \Gamma$, we define:
            $$
            P(\gamma): \gamma \text{ is a square}, \quad \text{ and } \quad Q(\gamma): \gamma \text{ is a rectangle}
            $$
            \begin{enumerate}[label=(\alph*)]
                \item $(\forall \gamma \in \Gamma), P(\gamma)$ is false because it reads as "For any polygon $\gamma \in \Gamma$, it is a square"; but a pentagon is a polygon and it is not a square.
                \item $(\exists \gamma \in \Gamma) : P(\gamma)$ reads as "There exists a polygon $\gamma \in \Gamma$, such that $\gamma$ is square" which is true because squares are polygons.
                \item $(\forall \gamma \in \Gamma), (P(\gamma) \implies Q(\gamma))$ is read as "For all polygons $\gamma \in \Gamma$, if $\gamma$ is a square then it is a rectangle" which is true because all squares are rectangles.
            \end{enumerate}

        \item Consider the set of birds $B$ and family of statements $\{P(b) : b \in B\}$ where $P(b)$ is "The bird $b$ can fly". This statement is false because there are birds that cannot fly. Therefore $(\forall b \in B), P(b)$ which says "For each bird $b$ in $B$ it can fly" is false. The negation $\neg ((\forall b \in B),P(b))$ must be true. The negation says "There is at least one bird $b$ such that the bird $b$ cannot fly" or $(\exists b \in B): \neg P(b)$. So we have the equivalence:
            $$
            \neg((\forall b \in B), P(b)) \equiv (\exists b \in B):\neg P(b)
            $$
            \label{ex:quantifiers4}

        \item Let $X$ and $Y$ be sets. For each $x \in X$, define $P(x)$ to be the statement $x \in Y$ and $Q$ be the statement "$X \cap Y \neq \varnothing$".
            \begin{enumerate}[label=(\alph*)]
                \item $Q$ says that there is at least one element in both $X$ and $Y$ and $Q \equiv (\exists x \in X): P(x)$.
                \item The negation of $Q$ namely $\neg Q$ is $X \cap Y = \varnothing$ and $\neg Q \equiv (\forall x \in X), \neg P(x)$.
            \end{enumerate}
            Thus, we have the equivalence: \label{ex:quantifiers5}
            $$
            (\forall x \in X), \neg P(x) \equiv \neg Q \equiv \neq((\exists x \in X) : P(x))
            $$
    \end{enumerate}
    \label{ex:quantifiers}
\end{example}

In examples \ref{ex:quantifiers}(\ref{ex:quantifiers4}) and (\ref{ex:quantifiers5}) the following rules hold:

\begin{enumerate}
    \item $\neg((\exists x \in X): P(x)) \equiv (\forall x \in X), \neg P(x)$
    \item $\neg((\forall x \in X), P(x)) \equiv (\exists x \in X): \neg P(x)$
\end{enumerate}

\noindent where the negation of a compound quantifier statement results in the flipping of the quantifier and negation of the specified statement. These are called De Morgan's laws in formal logic.

\begin{example}
    Define two sets $\Gamma$ and $\Delta$ where $\Gamma$ is the set of letters in Latin and $\Delta$ is the set of all words in \textit{The Oxford English Dictionary}.
    Define a mathematical statement that depends on two variables $(\gamma,\delta) \in \Gamma \times \Delta$ which says "The word $\delta$ begins with the letter $\gamma$" as $P(\gamma,\delta)$

    \begin{enumerate}
        \item For any fixed $\gamma \in \Gamma$, we can create $Q \equiv (\delta \in \Delta): P(\gamma,\delta)$. Varying $\gamma$, a family of mathematical statements $\{Q(\gamma):\gamma \in \Gamma\}$ which is parametrized by $\gamma \in \Gamma$. Therefore, we can append with a quantifier for $\gamma$ to create a mathematical statement:
            \begin{enumerate}[label=(\alph*)]
                \item The statement:
                    $$
                    (\forall \gamma \in \Gamma), Q(\gamma) \equiv (\forall \gamma \in \Gamma), (\exists \delta \in \Delta) : P(\gamma,\delta)
                    $$
                    \noindent which reads as "For every letter $\gamma \in \Gamma$, there is a word $\delta \in \Delta$ such that the word $\delta$ starts with the letter $\gamma$." This is true, because there are words starting with any letter in the alphabet in the dictionary.

                \item Another statement:
                    $$
                    (\forall \gamma \in \Gamma): Q(\gamma) \equiv (\exists \gamma \in \Gamma) : (\exists \delta \in \Delta) : P(\gamma,\delta)
                    $$
                    \noindent which reads as "There exists a letter $\gamma \in \Gamma$ such that there exists a word $\delta \in \Delta$ such that the word $\delta$ starts with the letter $\gamma$." This is also true, because for every letter there is a word that starts with that letter.

            \end{enumerate}

        \item Other combinations of quantifiers:
            \begin{enumerate}[label=(\alph*)]
                \item $(\forall \delta \in \Delta), (\exists \gamma \in \Gamma) : P(\gamma,\delta)$ reads as ``For every word $\delta \in \Delta$, there exists a letter $\gamma \in \Gamma$ such that the word $\delta$ starts with the letter $\gamma$.'' Which means every word starts with some letter which is true.

                \item $(\exists \delta \in \Delta) : (\forall \gamma \in \Gamma), P(\gamma,\delta)$ reads as ``There is a word $\delta \in \Delta$ such that for every letter $\gamma \in \Gamma$ the word starts with that letter''. This is saying that there is a word that starts with every letter in the alphabet which is false.

                \item $(\exists \delta \in \Delta) : (\exists \gamma \in \Gamma) : P(\gamma,\delta)$ reads as ``There exists a word such that there is a letter that the word starts with.'' This is true because every word starts with a letter.
            \end{enumerate}
    \end{enumerate}
    \label{ex:oxfordquantifier}
\end{example}

\begin{remark}
    Some remarks of Example \ref{ex:oxfordquantifier}.

    \begin{enumerate}
        \item We cannot generally move around quantifiers if there are many in a statement.

        \item However, switching two quantifiers of the same type that are adjacent is allowed. For example, $( \exists \gamma\in\Gamma ): (\exists \delta\in\Delta):P(\gamma,\delta)$ and $(\exists \delta\in\Delta) : (\exists \gamma\in\Gamma):P(\gamma,\delta)$ are the same and can be read as $\exists(\gamma\in\Gamma)\land(\delta\in\Delta) : P(\gamma,\delta) \equiv (\exists(\gamma,\delta)\in\Gamma\times\Delta):P(\gamma,\delta)$. So we have:
            \begin{align*}
                (\exists\gamma\in\Gamma):(\exists\delta\in\Delta):P(\gamma,\delta) &\equiv (\exists(\gamma,\delta)\in\Gamma\times\Delta):P(\gamma,\delta) \\
                &\equiv (\exists\delta\in\Delta):(\exists\gamma\in\Gamma):P(\gamma,\delta)
            \end{align*}

            Similarly,
            \begin{align*}
                (\forall\gamma\in\Gamma),(\forall \delta\in\Delta),P(\gamma,\delta) &\equiv (\forall(\gamma,\delta)\in\Gamma\times\Delta),P(\gamma,\delta) \\
                &\equiv (\forall \delta\in\Delta),(\forall\gamma\in\Gamma),P(\gamma,\delta)
            \end{align*}

        \item If there is more than one quantifier in a statement, they can be treated as nested statements. Allows to define negations more systematically. Recall that negation of a quantifier statement flips the quantifier and negates its statement. The following are equivalent and is the negation is done by moving inwards one quantifier at at time:
            \begin{align*}
                \neg((\exists\gamma\in\Gamma):(\exists\delta\in\Delta):P(\gamma,\delta)) &\equiv (\forall\gamma\in\Gamma),\neg((\exists\delta\in\Delta):P(\gamma,\delta)) \\
                &\equiv (\forall\gamma\in\Gamma),(\forall\delta\in\Delta),\neg P(\gamma,\delta)
            \end{align*}
            \label{rm:negquantifierremark}
    \end{enumerate}
    \label{rm:quantifierremark}
\end{remark}

\begin{example}
    $X$ and $Y$ are sets, consider two families of statements: $\{P(x):x \in X\}$ and $\{Q(y):y \in Y\}$.
    Suppose we have the statement $(\forall x \in X), (\exists y \in Y): P(x) \implies Q(y)$ and want to find negation.
    Using Remark \ref{rm:quantifierremark}(\ref{rm:negquantifierremark}):

    \begin{align*}
        \neg((\forall x \in X), (\exists y \in Y): P(x) \implies Q(y))
        &\equiv (\exists x \in X) : \neg((\exists y \in Y): P(x) \implies Q(y)) \\
        &\equiv (\exists x \in X) : (\forall y \in Y), \neg(P(x) \implies Q(y))
    \end{align*}

    Using the equivalence in Equation \ref{eq:negimplies}:

    \begin{align*}
        \neg((\forall x \in X), (\exists y \in Y): P(x) \implies Q(y))
        &\equiv (\exists x \in X) : (\forall y \in Y), \neg(P(x) \implies Q(y)) \\
        &\equiv (\exists x \in X) : (\forall y \in Y), P(x) \land (\neg Q(y))
    \end{align*}
\end{example}

The existential quantifier denotes that for at least one element $x \in X, P(x)$ is true, the following clarifies when a unique element holds the statement true.

\begin{definition}[Unique, Non-existential Quantifier]
    Let $X$ be a non-empty set and $\{P(x) : x \in X\}$ be a set of mathematical statements with domain $X$.

    \begin{enumerate}
        \item Unique Existential Quanifier: A unique existential quantifier is a symbol ``$(\exists! x \in X) :$'' where the statement $(\exists! x \in X) : P(x)$ is true when $P(x)$ is true for exactly one $x \in X$.

        \item Non-existential Quantifier: A non-existential quantifier is a symbol ``$(\nexists x \in X) :$'' where the statement $(\nexists x \in X) : P(x)$ is true when $P(x)$ is true for none of $x \in X$ (or, in other words, $P(x)$ is false $\forall x$).
    \end{enumerate}
\end{definition}

\begin{remark} Some remarks:

    \begin{enumerate}
        \item The symbol $\exists!$ reads as "there exists a unique" or "there exists exactly one".

        \item The symbol $\nexists$ reads as "there does not exist" or "there are no".

        \item Note that the non-existential quantifier is the negation of the existence quantifier. Therefore, the following equivalence holds

            $$
            (\nexists x \in X) : P(x) \equiv (\forall x \in X), \neg P(x) \equiv \neg((\exists x \in X) : P(x))
            $$
    \end{enumerate}
\end{remark}

\begin{example}
    Let $X$ be the set of planets in the solar system and $P(x)$ be the sentence "Humans can live on planet $x$" for $x \in X$.

    \begin{enumerate}
        \item The statement $(\exists! x \in X): P(x)$ reads as "There is only one planet in the solar system where humans can live", which is unknown because humans have not tried to live on any other planets, yet.

        \item The statement $(\nexists x \in X): P(x)$ reads as "There are no planets in the solar system which humans can live on" which is false because we live on the Earth which is in the solar system.
    \end{enumerate}
\end{example}
