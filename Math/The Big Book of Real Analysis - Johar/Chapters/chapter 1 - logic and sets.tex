\chapter{Logic and Sets}

\section{Introduction to Logic}

Mathematical statements (often called propositions) require proof to be determined if true or false conditional to some definitions or axioms accepted to be true.
Mathematical proofs require base level axioms as opposed to absolute truths.

\begin{remark}
    This is left empty
\end{remark}

\subsection{And, Or, Not}

Combinations of mathematical statements can be made or manipulated to create new ones.
Negation is done by writing the opposite of a statement.
With statements $P$ and $Q$, they can be combined with "and" or "or", "and" is called a logical conjunction and "or" is called a logical disjunction.
The combination of statements is called a compound statement whose truth can be deduced as well.

\begin{example}
    Consider
    $$
    P: \text{A is a vowel,} \quad \text{and} \quad Q: \text{B is a vowel}
    $$

    We know that $P$ is true and $Q$ is false.

    \begin{enumerate}
        \item Negating each statement results in
            $$
            \neg P: \text{A is not a vowel,} \quad \text{and} \quad \neg Q: \text{B is not a vowel}
            $$
            Which results in $\neg P$ being false and $\neg Q$ true. Negation switches the truth of a statement.
        \item Looking at "and" and "or":
            \begin{enumerate}[label=(\alph*)]
                \item The "and" connective is denoted with $\land$. $P \land Q$ says "A is a vowel and B is a vowel", this is false because both statements need to be true in order for the compound statement to be true.
                \item The "or" connective is denoted with $\lor$. $P \lor Q$ says "A is a vowel or B is a vowel (or both)". This is true because either one of the statements needs to be true or both.
            \end{enumerate}
    \end{enumerate}
\end{example}

\begin{remark}
    This is left empty
\end{remark}

\begin{example}
    Consider 
    $$
    P: \text{Lucy likes coffee} \quad \text{and} \quad Q: \text{Lucy likes tea}
    $$

\end{example}
