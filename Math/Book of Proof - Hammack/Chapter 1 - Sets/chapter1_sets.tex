\documentclass[10pt]{article}

\usepackage[utf8]{inputenc}
\usepackage{amsmath,amsthm,amssymb}
\usepackage{mathrsfs}
\usepackage{amsmath}
\usepackage{epsfig}
\usepackage{color}
\usepackage{indentfirst}
\usepackage{enumitem}
\usepackage[marginparwidth=1in]{geometry}
\usepackage{marginnote}
\usepackage{listings}
\usepackage{xcolor}
\usepackage{graphicx}
\usepackage{latexsym,amsfonts,amssymb,amsthm,amsmath}
\usepackage{hyperref}
\usepackage{cleveref}
\usepackage{matlab-prettifier}
\usepackage[
    backend=biber,
    style=alphabetic,
    sorting=ynt
]{biblatex}
\addbibresource{Reference.bib}

\newenvironment{theorem}[2][Theorem]{
    \begin{trivlist}
\item[\hskip \labelsep {\bfseries #1}\hskip \labelsep {\bfseries #2.}]}{
\end{trivlist}}
\newenvironment{lemma}[2][Lemma]{
    \begin{trivlist}
\item[\hskip \labelsep {\bfseries #1}\hskip \labelsep {\bfseries #2.}]}{
\end{trivlist}}
\newenvironment{exercise}[2][Exercise]{
    \begin{trivlist}
\item[\hskip \labelsep {\bfseries #1}\hskip \labelsep {\bfseries #2.}]}{
\end{trivlist}}
\newenvironment{problem}[2][Problem]{
    \begin{trivlist}
\item[\hskip \labelsep {\bfseries #1}\hskip \labelsep {\bfseries #2.}]}{
\end{trivlist}}
\newenvironment{question}[2][Question]{
    \begin{trivlist}
\item[\hskip \labelsep {\bfseries #1}\hskip \labelsep {\bfseries #2.}]}{
\end{trivlist}}
\newenvironment{corollary}[2][Corollary]{
    \begin{trivlist}
\item[\hskip \labelsep {\bfseries #1}\hskip \labelsep {\bfseries #2.}]}{
\end{trivlist}}

\newlist{indentlist}{enumerate}{2}
\setlist[indentlist, 1]{label=(\alph*), listparindent=\parindent}
\setlist[indentlist, 2]{label=\roman*., listparindent=\parindent}

\newlist{noindentlist}{enumerate}{2}
\setlist[noindentlist, 1]{label=(\alph*), listparindent=\parindent, leftmargin=6mm}
\setlist[noindentlist, 2]{label=\roman*., listparindent=\parindent}

\newenvironment{solution}{
\begin{proof}[Solution]}{
\end{proof}}

\newcommand{\rank}{{\rm rank\;}}
\newcommand{\image}{{\rm Im\;}}
\newcommand{\stack}{{\rm stack\;}}
\newcommand{\Span}{{\rm span\;}}
\newcommand{\diag}{{\rm diag\;}}
\newcommand{\col}{{\rm col\;}}
\newcommand{\cp}{{\rm cp\;}}
\newcommand{\R}{\mathbb{R}}
\newcommand{\N}{\mathbb{N}}
\newcommand{\Z}{\mathbb{Z}}
\newcommand{\Q}{\mathbb{Q}}
\theoremstyle{definition}
\newtheorem{remark}{Remark}
\newtheorem{proposition}{Proposition}
\newtheorem{property}{Property}
\newtheorem{conjecture}{Conjecture}
\newtheorem{assumption}{Assumption}
\newtheorem{example}{Example}
\newtheorem*{examplenull}{Example}
\newtheorem{definition}{Definition}
\newtheorem{fact}{Fact}
\newtheorem{exmp}{Example}[section]
\newtheorem{defn}{Definition}[section]
\def\n{{\bf n}}
\def\m{{\bf m}}
\def\r{{\bf r}}
\def\c{{\bf c}}
\def\scr#1{{\mathcal #1}}
\def\rep#1{(\ref{#1})}
\def\eq#1{
    \begin{equation}#1
\end{equation}}
\def\qed{ \rule{.1in}{.1in}}

\def\matt#1{
    \begin{bmatrix}#1
\end{bmatrix}}
\newcommand{\vecb}[1]{\boldsymbol{#1}}
\newcommand{\dvecb}[1]{\dot{\boldsymbol{#1}}}
\newcommand{\ddvecb}[1]{\ddot{\boldsymbol{#1}}}
\newcommand{\unitv}[1]{\hat{\boldsymbol{#1}}}
\newcommand{\vecbt}[1]{\Tilde{\boldsymbol{#1}}}
\newcommand{\vecbb}[1]{\Bar{\boldsymbol{#1}}}
\newcommand{\norm}[2]{\left\Vert {#1} \right\Vert_{#2}}
\newcommand{\tbf}[1]{\textbf{#1}}
\newenvironment{aequation}
{
    \begin{equation}
    \begin{aligned}}
        {
        \end{aligned}
\end{equation}}

\definecolor{codegreen}{rgb}{0,0.6,0}
\definecolor{codegray}{rgb}{0.5,0.5,0.5}
\definecolor{codepurple}{rgb}{0.58,0,0.82}
\definecolor{backcolour}{rgb}{0.95,0.95,0.92}
\lstdefinestyle{mystyle}{
    backgroundcolor=\color{backcolour},
    commentstyle=\color{codegreen},
    keywordstyle=\color{magenta},
    numberstyle=\tiny\color{codegray},
    stringstyle=\color{codepurple},
    basicstyle=\ttfamily\footnotesize,
    breakatwhitespace=false,
    breaklines=true,
    captionpos=b,
    keepspaces=true,
    numbers=left,
    numbersep=5pt,
    showspaces=false,
    showstringspaces=false,
    showtabs=false,
    tabsize=2,
}
\lstset{style=mystyle}

\begin{document}

% --------------------------------------------------------------
%                         Start here
% --------------------------------------------------------------

\title{Chapter 1 - Sets}
\author{Joseph Le}

\maketitle

\section{Introduction to Sets}

A \textbf{set} is a collection of \textbf{elements}, elements can be anything (numbers, points, functions, etc.). An \textbf{infinite} has an infinite number of elements, otherwise it is a \textbf{finite} set. Two sets are \textbf{equal} if they contain the same elements.

An element $a$ is represented as being \textbf{in} a set $A$ by $a \in A$ and an element $b$ is \textbf{not in} $A$ is shown as $b \notin A$

The set of \textbf{natural numbers}, the set of \textbf{integers}, and the \textbf{rational numbers} have reserved symbols

$$ \mathbb N = \{ 1,2,3,4,5,...\} $$

$$ \mathbb Z = \{ 0, \pm 1, \pm 2, \pm 3, ... \} $$

$$ \mathbb Q = \{ x : x = \frac{m}{n}, \text{ where } m,n \in \Z \text{ and } n \neq 0 \} $$

\noindent as well as the set of \textbf{real numbers}, $\mathbb R$.

For finite sets, the \textbf{cardinality} or \textbf{size} is the number of elements and is denoted as $|A|$, not to be confused with the absolute value of a number, this notation is used only for sets.

The \textbf{empty set} is unique and is a set with no elements usually denoted as $\varnothing$, $\varnothing$, or $\{\}$ and $|\varnothing| = 0$. A set containing the empty set is not empty as $M = {\varnothing}$ contains the empty set, and thus has a cardinality of 1. An analogy often used is thinking of sets as boxes containing things, these things can be other boxes, whether empty or not.

\textbf{Set-builder notation} is a special type of notation used to describe sets by giving its elements rules. Consider the even integers, it can be written as $E = \{0, \pm 2, \pm 4, \pm 6,...\} = \{2n: n \in \mathbb Z\} = \{ n : n \text{ is an even integer} \}= \{ n : n = 2k, k \in \Z \}$. The general format is $X = \{ \text{expression} : \text{rule} \}$

\begin{example}
    Examples of set-builder notation
    \begin{enumerate}
        \item $\{ n : n \text{ is a prime number} \}$
        \item $\{ n^2 : n \in \Z\} = \{ 0, 1, 4, 9, 16, \dots \}$
        \item $\{x \ in \R: x^2-2=0\}=\{ \pm \sqrt{2} \}$
    \end{enumerate}
\end{example}

\begin{example}
    Describe the set $A = \{ 7a + 3b: a,b \in \Z \}$

    \noindent \textbf{Solution:} $A$ contains all numbers of the form $7a + 3b$ where $a$ and $b$ are integers, for example $(a,b) = (0,0) \rightarrow 0$, etc. It can be determined that $A$ only contains integers, but which ones? Suppose $n \in \Z$, and $n = 7n + 3(-2n)$. Here $a = n$ and $b = -2n$, therefore $n \in A$, thus every integer is in $A$ and $A = \Z$.
\end{example}

Intervals on the number line are sets as well, for the real number line, these sets are infinite in size. These sets are given as any two numbers $a,b\in \R$ with $a<b$.

\begin{examplenull}
    These are examples of intervals
    \begin{itemize}
        \item Closed interval: $[a,b] = \{x \in \R : a \leq x \leq b\}$
        \item Open interval: $(a,b) = \{x \in \R : a < x < b\}$
        \item Half-open interval:
            \begin{itemize}
                \item $(a,b] = \{x \in \R : a < x \leq b\}$
                \item $[a,b) = \{x \in \R : a \leq x < b\}$
            \end{itemize}
        \item Infinite interval:
            \begin{itemize}
                \item $(a,\infty) = \{x \in \R: a < x\}$
                \item $[a,\infty) = \{x \in \R: a \leq x\}$
                \item $(-\infty,b) = \{x \in \R: x < b\}$
                \item $(-\infty,b] = \{x \in \R: x \leq b\}$
            \end{itemize}
    \end{itemize}
\end{examplenull}

\subsection*{Exercises}
\begin{enumerate}[label=\Alph*.]
    \item Write each of the following sets by listing their elements between braces
        \begin{enumerate}[label=\arabic*.]
            \item $\{5x-1: x\in\Z\} = \{-16, -11, -6, -1, 4, 9, 15\}$
                \stepcounter{enumii}
            \item $\{x \in \Z : -2\leq x < 7\} = \{-2, -1, 0, 1,2,3,4,5,6\}$
                \stepcounter{enumii}
            \item $\{x \in \R : x^2 = 3\} = \{\pm\sqrt{3}\}$
                \stepcounter{enumii}
            \item $\{x \in \R : x^2+5x=-6\}=\{2,3\}$
                \stepcounter{enumii}
            \item $\{x\in\R : \sin{\pi x} = 0\}=\{0,\pm2,\pm3,\pm4,\dots\}=\Z$
                \stepcounter{enumii}
            \item $\{x \in \Z : |x|<5\}=\{0,\pm1,\pm2,\pm3,\pm4\}$
                \stepcounter{enumii}
            \item $\{x\in\Z:|6x|<5\}=\{0\}$
                \stepcounter{enumii}
            \item $\{5a+2b:a,b\in\Z\}=\Z$
                \begin{proof}
                    If $a$ and $b$ are integers, then $5a+2b$ is also an integer since it is simple multiplication and addition. Then if $n = 5a+2b = 5n+2(-2n) = 5n -4n = n$ where $a = n$ and $b = -2n$, then $5a+2b$ can create any integer, thus the set is equal to the set of all integers, $\Z$. (not sure if this is even a valid proof, just copied the proof from example 1.2)
                \end{proof}
        \end{enumerate}
    \item Write each of the following sets in set-builder notation
        \begin{enumerate}[label=\arabic*.]
                \setcounter{enumii}{16}
            \item $\{2,4,8,16,13,64,\dots\} = \{2^x : x \in \N\}$
                \stepcounter{enumii}
            \item $\{\dots, -6,-3,0,3,6,9,12,\dots\} = \{3x: x \in \Z\}$
                \stepcounter{enumii}
            \item $\{0,1,4,9,16,25,36,\dots\}=\{x^2 : x \in \Z \}$
                \stepcounter{enumii}
            \item $\{3,4,5,6,7,8\} = \{x \in N: 2 < x < 9\}$
                \stepcounter{enumii}
            \item $\{\dots,\frac{1}{8},\frac{1}{4},\frac{1}{2},1,2,4,8,\dots\} = \{2^a: a \in \Z\}$
                \stepcounter{enumii}
            \item $\{\dots, -\pi,-\frac{\pi}{2},0,\frac{\pi}{2},\pi,\frac{3\pi}{2},2\pi,\dots\} = \{\frac{a\pi}{2}: a \in \Z \}$
        \end{enumerate}
    \item Find the following cardinalities
        \begin{enumerate}[label=\arabic*.]
                \setcounter{enumii}{28}
            \item $\left| \{\{1\}, \{2,\{3,4\}\},\varnothing\} \right| = 3$
                \stepcounter{enumii}
            \item $\lvert \{\{\{1\},\{2,\{3,4\}\}\varnothing\}\} \rvert = 1 $
                \stepcounter{enumii}
            \item $\lvert \{x \in \Z : |x| < 10\} \rvert = 19$
                \stepcounter{enumii}
            \item $\lvert \{x \in \Z : x^2 < 10\} \rvert = 7$
                \stepcounter{enumii}
            \item $\lvert \{x \in \N : x^2 < 0\} \rvert = 0$
        \end{enumerate}
    \item Sketch the following sets of points in the $x-y$ plane
        \begin{enumerate}[label=\arabic*.]
                \setcounter{enumii}{38}
            \item $\{(x,y) : x \in [1,2],y\in[1,2]\}$
                \stepcounter{enumii}
            \item $\{(x,y):x\in[-1,1],y=1\}$
                \stepcounter{enumii}
            \item $\{(x,y): |x|=2,y\in[0,1]\}$
                \stepcounter{enumii}
            \item $\{(x,y):x,y\in\R,x^2+y^2=1\}$
                \stepcounter{enumii}
            \item $\{(x,y):x,y\in\R,y\geq x^2-1\}$
                \stepcounter{enumii}
            \item $\{(x,x+k):x\in\R,k\in\Z\}$, this generates a set of parallel lines with slope $m=1$, because $(x,y) = (x,x+k) \rightarrow y=x+k$.
                \stepcounter{enumii}
            \item $\{(x,y)\in\R^2:(y-x)(y+x)=0\}$, here, either $(y-x)=0$ or $(y+x)=0$ or both. The first case gives $y=x$ a line through the origin with slope $m=0$, the second case gives $y=-x$ a line through the origin with slope $m=-1$, and when both are zero, this gives the intersection which is the origin.
                \begin{figure}
                    \centering
                    \includegraphics[width=0.75\linewidth]{images/exercise_1_1_D.jpg}
                    \caption{Exercises 1.1D}
                \end{figure}
        \end{enumerate}
\end{enumerate}

\section{The Cartesian Product}

Given two sets $A$ and $B$, "multiplying" them results a new set denoted as $A\times B$ and is called the \textbf{Cartesian product} of $A$ and $B$.

\begin{definition}
    An \textbf{ordered pair} is a list $(x,y)$ of two elements $x$ and $y$, enclosed in parentheses and separated by a comma.
\end{definition}

For example, $(2,4) \neq (4,2)$ because the order is different even though they contain the same elements. Like sets, the elements don't have to be just numbers.

\begin{definition}
    The \textbf{Cartesian product} of two sets $A$ and $B$ is another set, denoted as $A\times B$ defined as $A \times B = \{ (a,b) : a\in A, b\in B\}$.
\end{definition}

The elements of a Cartesian product are ordered pairs of elements from both sets. For example, if $A = \{k,l,m\}$ and $B = \{q,r\}$, then $A \times B = \{(k,q),(k,r),(l,q),(l,r),(m,q),(m,r)\}$.

\begin{fact}
    \label{cartcard}
    If $A$ and $B$ are finite sets, then $\left| A \times B\right| = \lvert A \rvert \cdot \lvert B \rvert$.
\end{fact}

\begin{example}
    Let $A = \{1,2,3,4,5,6\}$ be the set representing the sides of a dice. $|A \times A| = 6 \cdot 6 = 36$ by Fact \ref{cartcard}. The set $A \times A$ can be thought of the set of possible outcomes of rolling two dice, each result being an ordered pair.
\end{example}

$\R \times \R = \{(x,y):x,y\in \R\}$ is the set of points on the Cartesian plane. $\R \times \N$ is a set of points on lines parallel to the origin whose heights are natural numbers. $\N \times \N$ is the set of points on the Cartesian plane whose coordinates are only natural numbers (in the first quadrant).

Cartesian products can be made with other Cartesian products as well, for example $\R \times (\N \times \Z) = \{(x,(y,z)): x\in\R,(y,z)\in \N \times \Z\}$. \textbf{Ordered triples} are ordered lists of three elements, i.e. $(x,y,z)$. This can be extended to higher numbers as well, i.e. $A_1 \times A_2 \times A_3 \times \cdots \times A_n = \{x_1, x_2, x_3, \dots, x_n: x_i \in A_i \, \forall i \in \N\}$. A \textbf{Cartesian power} is repeated Cartesian products of a set with itself $n$ times ($n > 0 \in \Z$) and is denoted as $A_n = A \times A \times \cdots \times A = \{(x_1,x_2,\dots,x_n): x_i \in A\}$.

\begin{example}
    $S = \{H, T\}$ represents two sides of a coin. Tossing the coin 7 times can be represented with a Cartesian power $S^7$. An element may look like $(H,T,H,H,H,H,T)$. The cardinality of $\lvert S^7\rvert = 2^7$ as there are two options in seven places.
\end{example}

\subsection*{Exercises}
\begin{enumerate}[label=\Alph*.]
    \item Write out the indicated sets by listing their elements between braces.
        \begin{enumerate}[label=\arabic*.]
            \item Suppose $A = \{1,2,3,4\}$ and $B = \{a,c\}$
                \begin{enumerate}[label=(\alph*)]
                    \item $A \times B = \{(1,a),(1,c),(2,a),(2,c),(3,a),(3,c),(4,a),(4,c)\}$
                    \item $B \times A = \{(a,1),(a,2),(a,3),(a,4),(c,1),(c,2),(c,3),(c,4)\}$
                    \item $A \times A = \left\{
                            \begin{aligned}(1,1),(1,2),(1,3),(1,4),(2,1),(2,2),(2,3),(2,4),
                                \\(3,1),(3,2),(3,3),(3,4),(4,1),(4,2),(4,3),(4,4)
                        \end{aligned} \right\}$
                    \item $B \times B = \{(a,a),(a,c),(c,a),(c,c)\}$
                    \item $\varnothing \times B = \varnothing$
                    \item $(A\times B)\times B =\left\{
                            \begin{aligned}
                                &( (1,a), a),( ( 1,c ), a),( (2,a), a),( (2,c), a), \\
                                &( (3,a), a),( (3,c), a),( (4,a), a),( (4,c), a), \\
                                &( (1,a), c),( ( 1,c ), c),( (2,a), c),( (2,c), c),\\
                                &( (3,a), c),( (3,c), c),( (4,a), c),( (4,c), c) \\
                        \end{aligned}\right\}$
                    \item $A \times (B \times B) = \left\{
                            \begin{aligned}
                                &(1,(a,a)),(1,(a,c)),(1,(c,a)),(1,(c,c)),\\
                                &(2,(a,a)),(2,(a,c)),(2,(c,a)),(2,(c,c)),\\
                                &(3,(a,a)),(3,(a,c)),(3,(c,a)),(3,(c,c)),\\
                                &(4,(a,a)),(4,(a,c)),(4,(c,a)),(4,(c,c)),\\
                        \end{aligned}\right\}$
                    \item $B^3 = \left\{
                            \begin{aligned}
                                (a,a,a),(a,a,c),(a,c,c),(c,c,c),(c,a,a),(c,c,a),(a,c,a),(c,a,c)
                        \end{aligned}\right\}$
                \end{enumerate}
                \stepcounter{enumii}
            \item
                $
                \begin{aligned}[t]
                    \{x \in \R : x^2 = 2\} \times \{a,c,e\}  &= \{-\sqrt{2}, \sqrt{2}\} \times \{a,c,e\} \\
                    &= \{(-\sqrt{2},a),(-\sqrt{2},c),(-\sqrt{2},e),(\sqrt{2},a),(\sqrt{2},c),(\sqrt{2},e)\}
                \end{aligned}
                $
                \stepcounter{enumii}
            \item
                $
                \begin{aligned}[t]
                    \{x\in\R:x^2=2\}\times\{x\in\R:|x|=2\}  &= \{-\sqrt{2},\sqrt{2}\}\times\{-2,2\} \\
                    &= \{(-\sqrt{2},-2),(-\sqrt{2},2),(\sqrt{2},-2),(\sqrt{2},2)\}
                \end{aligned}
                $
                \stepcounter{enumii}
            \item
                $
                \begin{aligned}[t]
                    \{\varnothing\} \times \{0,\varnothing\} \times \{0,1\} &= \{(\varnothing, 0, 0),(\varnothing,\varnothing,0),(\varnothing,0,1),(\varnothing,\varnothing,1)\}
                \end{aligned}
                $
                \stepcounter{enumii}
        \end{enumerate}
    \item Sketch these Cartesian products on the $x-y$ plane $\R^2$ and $\R^3$.
        \begin{enumerate}[label=\arabic*.]
                \setcounter{enumii}{8}
            \item 9
                \stepcounter{enumii}
            \item 11
                \stepcounter{enumii}
            \item 13
                \stepcounter{enumii}
            \item 15
                \stepcounter{enumii}
            \item 17
                \stepcounter{enumii}
            \item 19
        \end{enumerate}
\end{enumerate}

\section{Subsets}
\begin{definition}
    If $A$ and $B$ are sets, and every element of $A$ is in $B$, then $A$ is a \textbf{subset} of $B$ and is denoted as $A \subseteq B$. Otherwise, $A \not\subseteq B$ if $A$ is \textit{not} a subset of $B$ which means that at least one element of $A$ is not an element of $B$
\end{definition}

\begin{example} These are all true
    \begin{enumerate}
        \item $\{2,3,7\} \subseteq \{2,3,4,5,6,7\}$
        \item $\{2,3,7\} \not\subseteq \{2,4,5,6,7\}$
        \item $\{2,3,7\} \subseteq \{2,3,7\}$
        \item $\{(x,\sin{x}: x\in\R\} \subseteq \R^2$
        \item $\{1,3,5,7,9,11,13,17,\dots\} \subseteq \N$
        \item $\N \subseteq \Z \subseteq \Q \subseteq \R$
        \item $\R \times \N \subseteq \R \times \R$
        \item $A \subseteq A, \; \forall A$
        \item $\varnothing \subseteq \varnothing$
    \end{enumerate}
\end{example}

The empty set $\varnothing$ is a subset of \textit{all} sets. This is because there are no elements in $\varnothing$ so there is not an element in $\varnothing$ that is not in $B$ or any other set. Formally written:

\begin{fact}
    The empty set is a subset of all sets, that is, $\varnothing \subseteq B$ for any set $B$.
\end{fact}





\section{Power Sets}

\section{Union, Intersection, Difference}

\section{Complement}

\section{Venn Diagrams}

\section{Indexed Sets}

\section{Sets That Are Number Systems}

\section{Russell's Paradox}

% \printbibliography
\end{document}
